%%%%%%%%%%%%%%%%%%%%%%%%%%%%%%%%%%%%%%%%%%%%%%%%%%%%%%%%%%%%%%%
%
% Welcome to Overleaf --- just edit your LaTeX on the left,
% and we'll compile it for you on the right. If you open the
% 'Share' menu, you can invite other users to edit at the same
% time. See www.overleaf.com/learn for more info. Enjoy!
%
%%%%%%%%%%%%%%%%%%%%%%%%%%%%%%%%%%%%%%%%%%%%%%%%%%%%%%%%%%%%%%%


% Inbuilt themes in beamer
\documentclass{beamer}
\usepackage{amsmath}
% Theme choice:
\usetheme{CambridgeUS}
\usepackage{listings}
\usepackage{blkarray}
\usepackage{listings}
\usepackage{subcaption}
\usepackage{url}
\usepackage{tikz}
\usepackage{tkz-euclide} % loads  TikZ and tkz-base
%\usetkzobj{all}
\usetikzlibrary{calc,math}
\usepackage{float}
\renewcommand{\vec}[1]{\mathbf{#1}}
\usepackage[export]{adjustbox}
\usepackage[utf8]{inputenc}
\usepackage{amsmath}
\usepackage{amsfonts}
\usepackage{tikz}
\usepackage{hyperref}
\usepackage{bm}
\usetikzlibrary{automata, positioning}
\providecommand{\pr}[1]{\ensuremath{\Pr\left(#1\right)}}
\providecommand{\mbf}{\mathbf}
\providecommand{\qfunc}[1]{\ensuremath{Q\left(#1\right)}}
\providecommand{\sbrak}[1]{\ensuremath{{}\left[#1\right]}}
\providecommand{\lsbrak}[1]{\ensuremath{{}\left[#1\right.}}
\providecommand{\rsbrak}[1]{\ensuremath{{}\left.#1\right]}}
\providecommand{\brak}[1]{\ensuremath{\left(#1\right)}}
\providecommand{\lbrak}[1]{\ensuremath{\left(#1\right.}}
\providecommand{\rbrak}[1]{\ensuremath{\left.#1\right)}}
\providecommand{\cbrak}[1]{\ensuremath{\left\{#1\right\}}}
\providecommand{\lcbrak}[1]{\ensuremath{\left\{#1\right.}}
\providecommand{\rcbrak}[1]{\ensuremath{\left.#1\right\}}}
\providecommand{\abs}[1]{\vert#1\vert}

\newcounter{saveenumi}
\newcommand{\seti}{\setcounter{saveenumi}{\value{enumi}}}
\newcommand{\conti}{\setcounter{enumi}{\value{saveenumi}}}
\usepackage{amsmath}
% Title page details: 
\title{Assignment6} 
\author[CS21BTECH11017]{G HARSHA VARDHAN REDDY ( CS21BTECH11017 )}
\date{\today}
\logo{}


\begin{document}
% Title page frame
\begin{frame}
    \titlepage 
\end{frame}
% Outline frame
\begin{frame}{Outline}
    \tableofcontents
\end{frame}


% Lists frame
\section{Abstract}
\begin{frame}{Abstract}
This document contains $6^{th}$ problem from exercise $2$ of CBSE Class $12$ (Probability) .
\end{frame}
\section{Problem}
\begin{frame}{Problem}

\textbf{Exercise $2$ Problem $3$ }Let $E$ and $F$ be events with $P(E)=\frac{3}{5}$ , $ P(F)=\frac{3}{10}$ and $P(EF)=\frac{1}{5}.$ Are
$E$ and $F$ independent?
\end{frame}

\section{Theory}
%\subsection{Independent Events}
\begin{frame}{Theory}
\begin{block}{Independent events :}
Two events are independent if the incidence of one event does not affect the probability of the other event.(or)\\
Two events $A,B$(say) are said to be independent if $P(A|B)=P(A) $
\begin{align}
    &\implies P(A|B)=\frac{P(AB)}{P(B)}=P(A)\\
    &\implies P(A)\times P(B)=P(AB)
\end{align}
\end{block}
%\textbf{\large{\textcolor{cyan}{Independent events :}}}
\end{frame} 
% Blocks frame
\section{Solution}
\begin{frame}{Solution}
Let's denote the outcome of the experiment by a random variable $X\in \cbrak{0, 1}$, where $X = 0$ denotes occurrence of event $E$ and $X = 1$ denotes occurrence of event $F$.
\begin{align}
    \implies P(X=0)&=\frac{3}{5},\label{eq:1}\\
    P(X=1)&=\frac{3}{10}\label{eq:2} \text{ and}\\
    P(X=0,X=1)&=\frac{1}{5}\label{eq:3}
\end{align}
\end{frame}
\begin{frame}{}
Let's check whether the above events are independent or not.\\
From \eqref{eq:1},\eqref{eq:2}
\begin{align}
    P(X=0)\times P(X=1)=\frac{3}{5} \times \frac{3}{10}\\
    \implies P(X=0)\times P(X=1)=\frac{9}{50}\label{eq:7}
\end{align}
From \eqref{eq:3} and \eqref{eq:7} it's clear that
\begin{align}
    P(X=0,X=1)&\neq P(X=0)\times P(X=1)
\end{align}
Which says that the events $ E$ and $F$ are not independent.

\end{frame} 

\end{document}